\documentclass[12pt]{exam}
\usepackage{gensymb}
\usepackage{graphicx}
\usepackage{float}
\author{saikiran}
\begin{document}
\begin{questions}

\question 
	In Figure \ref{fig1},if tangents PA and PB from an external point P to a circle with center O,are inclined to each other at angle of 80\degree,then \angle AOB is eqaul to
\begin{figure}[H]
\centering
\includegraphics[width=\columnwidth]{/storage/emulated/0/Latex/B5.png}
\caption{}
\label{fig1}
\end{figure}
\begin{enumerate}
	\item $100\degree$
	\item $60\degree$
	\item $80\degree$
	\item $50\degree$
\end{enumerate}

\question 
Two concentric cicrles are of radii 4cm and 3cm.Find the length of the chord of the larger cicrle which touches the smaller cicle.

\question
	In figure \ref{fig2},a triangle ABC \angle B=90 \degree is shown.Taking AB as diameter,a cicrle has benn drawn intersecting AC at point P.Prove that  the tanget drawn at point P bisects BC
\begin{figure}[H]
\centering
\includegraphics[width=\columnwidth]{/storage/emulated/0/Latex/C11.png}
	\caption{}
	\label{fig2}
\end{figure}
\question 
Prove that a parallelogram circumscribing a circle is rhombus.

\question 
\begin{parts}

	\part In Figure \ref{fig3},two circles with centres at O and O' of of radii 2r and r respectivily,touch each other internally at A.A chord AB of the bigger circle meets the smaller circle at C,Show that C bisects AB.
\begin{figure}[H]
\centering
\includegraphics[width=\columnwidth]{/storage/emulated/0/Latex/E12a.png}
\caption{}7
	\label{fig3}
\end{figure}

	\part In Figure \ref{fig4}, O is center of a circle of radius 5cm. PA and BC are tangents to the circle at A and B respectively.If OP=13cm, then find the lenght of PA and BC 
\begin{figure}[H]
\centering
\includegraphics[width=\columnwidth]{/storage/emulated/0/Latex/E12b.png}
\caption{}
	\label{fig4}
\end{figure}
\end{parts}

\question 
In two concentric circles, a chord of length 48 cm of the larger circle is a tangent to the smaller circle, whose radius is 7 cm. Find the radius of the larger circle.
\\

\question
\begin{parts}
\part If two circles touch each other externally, then prove that the point of contact lies on the line joining their centres.

\part Prove that the lengths of two tangents drawn from an
external point to a circle are equal.
\end{parts}
\end{questions}
\end{document}
